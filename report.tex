% !TEX program = xelatex
\documentclass[12pt,a4paper]{report}
% \usepackage[utf8]{inputenc}
\usepackage{lmodern}
\usepackage{fontspec}
\usepackage{url}
\usepackage{indentfirst}
\usepackage{fancyhdr}
\usepackage{hyperref}
\hypersetup{
    colorlinks,
    citecolor=black,
    filecolor=black,
    linkcolor=black,
    urlcolor=black
}

\renewcommand\UrlFont{\ttfamilylatin}
\usepackage{xltxtra} 

\XeTeXlinebreaklocale "th"
\XeTeXlinebreakskip = 0pt plus 1pt   %


\usepackage{titling}
\usepackage{titlesec}
\usepackage[top=0.5in, bottom=0.5in, left=1in, right=1in]{geometry}
\newfontfamily\thaifont[Script=Thai]{Laksaman}

\setmainfont{Laksaman}
\usepackage{polyglossia}

\setdefaultlanguage{thai}

\usepackage{graphicx}
\usepackage{fancyhdr}

\title{การซ่อมและบำรุงรักษาคอมพิวเตอร์}
\author{นายชนกันต์ มุ่งถิ่น}
\date{วันที่}

\renewcommand{\normalsize}{\fontsize{12pt}{24pt}\selectfont}
\titleformat{\chapter}[display]{\normalfont\huge\bfseries\centering}{\chaptertitlename \thechapter}{12pt}{\LARGE}
\titleformat{\section}
  {\normalfont\fontsize{12}{15}\bfseries}{\thesection}{1em}{}

\begin{document}

\begin{titlepage}
\begin{center}

\pagenumbering{gobble}


{\Large \textbf{รายงาน}}\\[0.5cm]
{\Large \textbf{เรื่อง \thetitle}}\\[2cm]
{\Large \textbf{จัดทำโดย}}\\[0.5cm]
{\Large \textbf{\theauthor}}\\[0.2cm]
{\Large \textbf{ชั้น ปวช.2/1 เลขที่ 10}}\\[2cm]
{\Large \textbf{เสนอ}}\\[0.5cm]
{\Large \textbf{ครูเอกชัย สบายจิตร}}\\[1.8cm]
{\Large \textbf{รายงานนี้เป็นส่วนหนึ่งของรายวิชา\\[0.5cm] การซ่อมและบำรุงรักษาคอมพิวเตอร์}}\\[1cm]
{\Large \textbf{ภาคเรียนที่ 2 ปีการศึกษา 2565}}\\[0.5cm]
{\Large \textbf{โรงเรียนจิตรลดาวิชาชีพ สถาบันเทคโนโลยีจิตรลดา}}

\end{center}
\end{titlepage}


\chapter{บทนำ}

\section{หลักการและเหตุผล}
เวลาที่เจอปัญหาร่างกายต่างๆ อย่างเช่น นอนตื่นมาแล้วพบว่า เท้ามีอาการบวมแดง
และมีแผลพุพอง ทำให้ต้องไปโรงพยาบาล ซึ่งบางครั้งนั้น ก็เป็นไปได้อย่างยากลำบากมาก
ทั้งเรื่องเส้นทาง และเวลาในการรอพบหมอ ที่บางครั้งนั้นจำเป็นต้องมีการรอ 3-4 ชั่วโมง แล้วคุยเพียงแค่ 10 นาที เพื่อที่จะได้คำตอบมาว่า "อ๋อ คุณโดนแมลงก้นกระดกกัด กินยาแก้อักเสบแผงเดียวก็หายแล้ว" โดยเป็นปัญหาที่มักเกิดจากจำนวนคนในโรงพยาบาลที่เยอะมาก ในขณะที่หมอที่รับผิดชอบนั้นมีจำนวนน้อย
โดยถ้าหากรู้ว่า ไม่มีความจำเป็นจะต้องมาถึงโรงพยาบาล ก็คงจะประหยัดเวลาไปได้เยอะพอสมควร

ในปัจจุบันนั้น ยังคงเป็นปัญหาที่้เกิดขึ้นบ่อยครัง ภายในโรงพยาบาลที่มีผู้รับผิดชอบไม่เพียงพอ หรือคลินิกเล็กๆ อย่างการที่คลินิกมีมักมีผู้รับผิดชอบน้อย อาทิเช่น หนึ่งหรือสองคน จีงทำให้เป็นปัญหาเมื่อมีผู้คนที่มีความต้องการจะพบหมอ เพื่อปรึกษา หรือสอบถามอาการที่พบเจออยู่

โดยได้สังเกตว่า ถ้ามีแพลตฟอร์มที่เราสามารถติดต่อกับหมอได้โดยตรง หรือค้นหาโรงพยาบาล หรือคลินิก ที่ใกล้เราได้อย่างง่ายดาย โดยที่มีข้อมูลเกี่ยวกับคลินิก หรือโรงพยาบาลนั้นๆ ว่าเปิดอยู่หรือไม่ คลินิกนั้นๆ รับรักษาด้านอะไรบ้าง รวมถึงระบบที่เราสามารถนัดจองผ่านเว็บไซต์ได้เลย โดยไม่จำเป็นต้องมีการเดินทางไปยังคลินิก หรือโรงพยาบาล หรือการโทรสอบถาม ที่จำเป็นจะต้องมีการสอบถามข้อมูลคลินิก หรือโรงพยาบาล อย่างมากมาย อาทิเช่น เวลาเปิด หรือ รับรักษาอะไรบ้าง เป็นต้น

ด้วยเหตุผลดังกล่าวนี้ จึงมีแนวคิดในการจัดทำโครงการหมอแมพ โดยเป็นเว็บไซต์ที่อนุญาตให้ผู้คนที่ต้องการค้นหาโรงพยาบาลที่ใกล้ตัว และรู้สถานะของโรงพยาบาลได้ทันที พร้อมกับสามารถนัดคุยกันได้เลย เพื่อที่จะลดความแออัดภายในโรงพยาบาลลง
และทางฝั่งโรงพยาบาลก็สามารถที่จะตั้งสถานะตัวเอง ว่าเปิดหรือปิด รวมถึงสามารถบอกรายละเอียดต่างๆเกี่ยวกับโรงพยาบาลได้ทันที



\nocite{*}
\bibliographystyle{plain}
\bibliography{bibliography}
\end{document}

